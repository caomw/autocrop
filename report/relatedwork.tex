%% ----------------------------------------------------------------------------
% BIWI SA/MA thesis template
%
% Created 09/29/2006 by Andreas Ess
% Extended 13/02/2009 by Jan Lesniak - jlesniak@vision.ee.ethz.ch
%% ----------------------------------------------------------------------------
\newpage
\chapter{Related Work}

As mentioned in the previous section, the aim is to improve the experience of
viewing photos from a mobile phone gallery as wallpapers.
This requires the determination of whether an image should be selected for
display, and the selection of a window or crop of the original image which
should be displayed on a given screen.

There is no prior work for automatically selecting wallpapers.
The most similar work is the summarisation of photo albums
\cite{sinha2009personal}.
Unfortunately, there is lack of detail and reliance on detailed
annotations.
Thus in this study, the problem of deciding if a photo could be used as a
wallpaper is dealt with in a simple manner which requires minimal annotations.

Depending on which object or scene is given focus, an image can be perceived
very differently by a viewer.
Thefore there is great interest in attempting to improve how well an image is
composed.
To this effect, there have been many previous studies in cropping or deforming
images to a target size or aspect ratio.

Seam carving is an effective way to removing less
interesting seams or regions in an image but unfortunately can warp the image
badly when failing to work successfully \cite{avidan2007seam}.
Other similar methods such as Multi-operator retargeting also suffer from
heavy deformation and artifacts depending on the input image \cite{rubinstein2009multi}.
In fact, \cite{rubinstein2010comparative} notes that:
\emph{"Cropping, although a relatively naive operation, is still one of the most
	favored methods, most often since it does not create any artifacts. Our
	findings indicate that the search for an optimal cropping window, which
	was somewhat abandoned by researchers in the past few years, could often
	be favorable and should not be overlooked."}
Therefore, cropping algorithms are focused on for choosing how to display a
candidate wallpaper image on a given display.

The automatic cropping of an image can be done in two major ways, one which
requires a set of rules, and a learning-based model which associates a set of
features to a score.
Rules-based croppers such as \cite{liu2010optimizing, zhang2005auto} can suffer
from bias or imprecision due to human-selected rules and parameters.
Learning-based croppers such as \cite{park2012modeling, yan2013learning} are
not without fault however, with low precision due to lack of training data being
a particularly big issue.

A novel method suggested by Fang et al. \cite{fang2014automatic} attempts to
combine the advantages of both approaches by introducing three cues into the
learning and cropping stages.
These include saliency composition, boundary simplicity, and content
preservation.
This is described in greater detail in the next sections.
Another improvement in the suggested method is the use of public datasets such
as those which can be found on image hosting services such as Flickr and
Photo.net.
These services can indicate how good an image is perceived to be and allow for
automatic annotation of possible image crops.

