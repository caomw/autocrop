%% ----------------------------------------------------------------------------
% BIWI SA/MA thesis template
%
% Created 09/29/2006 by Andreas Ess
% Extended 13/02/2009 by Jan Lesniak - jlesniak@vision.ee.ethz.ch
%% ----------------------------------------------------------------------------


\chapter{Introduction}

The widespread use of mobile computing devices such as smartphones and tablet
computers has led to the creation of large personal photo collections.
Unlike previous methods, photo capture using modern smartphones has a low cost.
There are low space restrictions and photos are simple to take with modern
applications.
This leads to lower inhibition in taking photos, and a wider variety photos in
less carefully curated collections.
These collections can include photos capturing among other objects:

\begin{itemize}
\item Natural scenes
\item Cityscapes
\item Quick notes (street signs, maps, documents)
\item Screenshots
\item Quick shots for instant messaging
\end{itemize}

It may sometimes be desirable to review some photos from a mobile phone via the use of widgets or wallpaper carousels.
One such application which serves this purpose is Muzei for Android \footnote{\url{http://www.muzei.co/}}.
A user can set Muzei to show a random photo from the mobile phone gallery as a wallpaper with a new photo being chosen at a set time interval.
It should be noted that such a photo carousel may suffer from two issues.

\begin{figure}
\centering
\begin{subfigure}{0.46\columnwidth}
  \centering
  \includegraphics[width=0.95\columnwidth]{../figures/Michael/2013-12-04 13.44.24.jpg}
  \caption{An academic poster, useful for review purposes but not possibly unsuitable as a wallpaper.}
\end{subfigure}
\hfill
\begin{subfigure}{0.46\columnwidth}
  \centering
  \includegraphics[width=0.95\columnwidth]{../figures/Michael/2013-12-25 16.32.32.jpg}
  \caption{Scene of a beach at a vacation destination. Possibly suitable as a wallpaper.}
\end{subfigure}
\caption{An example of a potentially suitable and unsuitable photo to use in creating a wallpaper for a mobile device.\label{fig:good_n_bad_wallpaper}}
\end{figure}

The first issue is that certain photos may not be suitable to be used as a wallpaper.
For instance, a bus route timetable may be useful to have for the cases where an estimated travel time is required, but not the most aesthetically pleasing image to have on the background of a mobile phone.
An example of this is illustrated in figure \ref{fig:good_n_bad_wallpaper}
At the most basic level, the objects present in the image could be used to determine if it would be suitable.

The second issue comes from the alignment or cropping of a photo.
Since a photo must be formatted appropriately to be shown on the given display, some information has to be discarded.
A naive approach of centering the photo may lead to the cutting or complete omittance of objects.
It would be desirable to crop a given image to a target aspect ratio without losing interesting regions.
To this effect, work from \ref{fang2014automatic} is expanded on to provide an effective automatic cropping algorithm.

This project aims to address the two mentioned issues: determination of wallpaper suitability and image cropping to fit a target display in an aesthetically pleasing manner.
In short form, we will refer to the two areas as \emph{Selection} and \emph{Cropping}.

To solve the photo selection problem, we propose a simple algorithm based on object class recognition and machine learning in section \ref{sec:meth_selection}.
Out of a collection of images which have been annotated by 4 individuals, a subset is selected where annotations have consensus.
The annotations are for whether the corresponding image should be used as a wallpaper.
The final selection algorithm performs well on this dataset with an error of just 3.7\%.

The cropping of images is carried out using a saliency and learning based model which derives from the algorithm introduced in \cite{fang2014automatic}.
We extend the algorithm by considering boundary simplicity conditions for each image edge and employing a shrinking heuristic for evaluating potential crop regions.
This work is outlined in section \ref{sec:method_cropping}.
Our algorithm outperforms the approach of \cite{fang2014automatic} by approximately 6\%, achieving a maximum overlap of $0.782 \pm 0.004$ with crop regions annotated by professionals via the Amazon Mechanical Turk platform.
The annotations are sourced from \cite{fang2014automatic}.

In the following sections, We introduce the methodology and used datasets for both wallpaper selection and automatic cropping, and evaluate the performance of the two algorithms in a quantitative and qualitative manner.

