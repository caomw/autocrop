%% ----------------------------------------------------------------------------
% BIWI SA/MA thesis template
%
% Created 09/29/2006 by Andreas Ess
% Extended 13/02/2009 by Jan Lesniak - jlesniak@vision.ee.ethz.ch
%% ----------------------------------------------------------------------------


\chapter{Introduction}

The widespread use of mobile computing devices such as smartphones and tablet
computers has led to the creation of large personal photo collections.
Unlike previous methods, photo capture using modern smartphones has a low cost.
There are low space restrictions and photos are simple to take with modern
applications.
This leads to lower inhibition in taking photos, and a wider variety photos in
less carefully curated collections.
These collections can include photos capturing among other objects:

\begin{itemize}
\item Natural scenes
\item Cityscapes
\item Quick notes (street signs, maps, documents)
\item Screenshots
\item Quick shots for instant messaging
\end{itemize}

Not all photos taken are of high value to a user.
For example, photos of posters, street signs or maps are used for short term
memory and do not have value in being reviewed.
Additionally, some photos are intentionally taken with less care for use in
instant messaging.

Photos which are taken with care can often have high value in being reviewed
occasionally.
To this purpose, applications such as Muzei for Android allow the displaying of
photos from a mobile phone album as a wallpaper.
Unfortunately, this is not done in the most ideal manner.
There are two big issues, (1) not all images are appropriate to be used as a
wallpaper, and (2) the image is not well aligned when displayed.
For example, it is undesirable to have a passport scan as a wallpaper or an
image where only the rear of an elephant is visible.

This project aims to improve this experience and other similar experiences by
first selecting photos which are appropriate to be used, then cropping the photo
appropriately to display on the screen of a mobile phone.
This is to be done in a fully automatic manner where no user intervention is
required.

